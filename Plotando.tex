\documentclass[12pt,a4paper]{article}
\usepackage[utf8]{inputenc}
\usepackage[T1]{fontenc}
\usepackage[portuguese]{babel}
\usepackage{amsmath}
\usepackage{amsfonts}
\usepackage{amssymb}
\usepackage{makeidx}
\usepackage{graphicx} % essencial
\usepackage{tikz} % essencial 
\usepackage{pgfplots} % essencial
\usepackage{pgfplotstable} % essencial
\usepackage{times} % Somente a fonte
\usepgfplotslibrary{dateplot} % Data como eixo x
\usepackage[left=2.00cm, right=2.00cm, top=2.00cm, bottom=2.00cm]{geometry} % margem
\author{Eduardo Adame}
\title{Como plotar gráficos}
\pgfplotsset{width=.8\textwidth,compat=1.9} %tamanho
\begin{document}
	\begin{figure}[!h]
		\centering %centraliza
		\begin{tikzpicture} %inicia o ambiente tikz
		\begin{axis}[ %inicia o eixo
		title = COVID-19 no estado da Ceará, %titulo
		date coordinates in=x, % data em x
		xticklabel style={rotate=90,anchor=near xticklabel},
		xticklabel={\day/\month}, % data no formato dia/mes
		xlabel={Tempo}, % Descrição eixo x
		xlabel style={yshift=-1pt},
		ylabel=Quantidade de Casos, % Descrição eixo y
		ymin=0, % mínimo y
		ymax=8000,  % máximo y
		xmin=2020-03-17, % mínimo x
		xmax=2020-04-30, % máximo x
		ylabel style={yshift=10pt},
		legend style={
		at={(0.5,-0.2)},anchor=north,legend columns=-1,
		}, %posição das legendas
		], %fim das propriedades dos eixos
		\addplot table [x index=0,y index=1] {Dados Brasil/DATA-CE.dat}; %Dados na subpasta
		\addlegendentry{Contaminados} %legenda
		\addplot table [x index=0,y index=2] {DATA-CE.dat};%Dados no mesmo diretório
		\addlegendentry{Óbitos}
		\end{axis}
		\end{tikzpicture} 
	\end{figure}
\end{document}